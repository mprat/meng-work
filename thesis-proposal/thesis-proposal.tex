\documentclass[12pt]{article}
\begin{document}

\title{Video Parsing for Educational Videos}
\author{Michele Pratusevich}
\date{\today}
\maketitle

\section{Abstract}

The amount of educational video content on the internet is increasing at an enormous rate, because of the increase in the number of massively open online courses (MOOCs). The intents of learners watching these videos are many-fold, and it is often hard to manually create and curate videos that are flexible for all users and applications. The purpose of this thesis is to create a descritor of scenes from educational videos. With a vision-based non-aural information extractor, a computer will produce a textual description of scenes. [todo] In a structured way that will allow creators of user-facing applicatons flexibility in their user interface design. The problems of optimal content presentation and specific context-dependent class examples will be easier and more flexible given a human and machine-readable automatic understanding of a video. Examples of front-end applications can be intra-video search for concepts using words instead of scrubbing, an example-extractor from a lecture video, or an automatic tool for adapting classroom videos into a MOOC format. This thesis will present an automated video parsing, demonstrate its use in one or two applications, test the applications, and demonstrate the effectiveness of the video parsing by easy incorporation of user feedback.

\section{Introduction}
\section{Related work}

\subsection{How Video Production Affects Student Engagement}

\begin{itemize}
\item Categorize videos into 6 types
\item table of recommendations for MOOC video creators -- I can use this as a template for starting to make metrics
\item only looks at math / science courses
\item proxies for engagement were engagement time and problem attempt
\item manually looked through each video to categorize the type.
\item types: lecture, tutorial, other. analysis only focuses on lectures and tutorials (a tutorial is something like a problem-solving walkthrough)
\item manually looked through each video and coded production style. 
\item production styles: slides, code, khan-style, classroom, studio, office desk
\item each video could have multiple labels. 
\item seems like labels were assigned not on a per-segment but rather on a per-video basis, with a video containing multiple labels
\item optimal engagement is videos under 6 minutes
\item interspersing talking heads with slides is more engaging than just slides
\item editing out filler speech in post-production is useful for student engagement
\item lectures should have a first-time watching experience; tutorials should support re-watching and skimming
\end{itemize}

\rightarrow the big takeaway for me is the 6-type parsing, some of the recommendations give indicators of how a parsing could be used in the future, and the parsing around ``lecture'' and ``tutorial'' is vague enough that more work can be done in figuring out a more optimal parsing

\section{Work already done}
\section{Planned work}
\section{Timeline}
\section{Testing}

\end{document}